%--------致谢
\begin{acknowledgements}
	矿大七年的求学经历即将结束,纵有千万般不舍,也要走向社会,用自己的在矿大学到的知识回报这个社会。作为矿大
	新百年的学子,矿大「好学力行、求是创新、艰苦奋斗、自强不息」的校训将成为我人生中的座右铭,在接下来
	的人生中,矿大精神将永远激励我前进,在这再次感谢矿大对我培养。
	
	首先感谢我的导师高井祥老师和孙久运老师,您们是我研究生生涯中的指明灯,指引我走向学术生涯道路。两位老师孜孜不倦的精神使
	我受益终身,你们为我提供了优越的科研环境,耐心和用心地指导我们开展学术活动,严谨的科研思维
	和扎实的工作作风是我的楷模。当我申请去中科院电子所苏州研究院实习,你们坚定的支持是我最大的动力,再次感谢你们。
	
	感谢论文中所有参考文献的作者,是你们杰出的工作,让我产生一点点灵感,你们是巨人的肩膀。还要感谢\LaTeX
	中文社区的肖立顺师兄,你无偿提供的矿大毕业论文\LaTeX{}模板,使我能够使用优美的\LaTeX{}完成论文排版工作。
	
	也要感谢远在家乡的家人,是你们坚定不移的支持,是我读研的全部动力,这份理解和付出无法报答,希望你们身体健康,永远
	幸福。
	
	最后,感谢各位专家和老师,在百忙之中评阅本论文并提出宝贵的意见和建议,我会聆听你们的教导继续努力。
\end{acknowledgements}

%--------中文摘要
\begin{cabstract}
	% 信息是当代社会重要的决策资源,随着移动互联网的迅速发展,以新浪微博为代表的
	% 社交空间数据呈现爆炸式增长,然而这些数据存在异构性、不规则性和海量性等
	% 特点,使得空间数据分析、空间数据挖掘和信息提取愈发难以处理。
	% 随着移动互联网的迅速发展,以新浪微博为代表的社交网络数据呈现爆炸式增长,其中
	% 包含的空间数据更值得关注和研究。然而这些数据存在着异质性、不规则性和海量性等
	% 特点,使得空间数据查询、空间数据挖掘和空间知识提取愈发难以处理。
	
	数字城市是智慧城市重要的组成部分,也同时面临着海量空间数据获取、管理、分析和挖掘等挑战。
	移动互联网的发展使得网络空间数据呈现爆炸式增长,其中蕴含的信息对智慧城市建设有着重要的
	参考建议,然而这些数据存在着异质性、不规则性和海量性等特点,使得空间数据查询、空间数据挖掘和
	空间知识提取愈发难以处理。
	
	传统的空间分析工具面对上述需求往往捉襟见肘,本文对当前流行的并行计算框架Spark进行空间扩展,
	构建Spatial-Spark并行空间计算框架。以此为基础,对海量新浪微博POI进行同位模式
	挖掘,对全国新浪微博用户空间位置进行人口网络图分析,本文所作的工作和结论如下:
	
	(1)对Spark RDD((Resilient Distributed Datasets)进行空间维度上扩展,对点、线和面构建了相应的Spatial RDD,
	支持海量空间数据读写、空间坐标转换和分区空间数据索引。提供空间拓扑查询、空间K邻居查询和空间连接查询三个常用的
	空间查询模块,通过搭建Hadoop/Spark计算集群,验证了Spatial-Spark在处理海量空间数据方面的优势。
	
	(2)使用新浪微博API获取全国范围内微博POI数据,对其进行同位模式挖掘。首先分析同位模式
	挖掘算法的关键,使用Spatial-Spark对全连接算法进行并行化设计。对上海、武汉和重庆三市二阶模式
	进行比较,不同城市呈现不同模式;选择距离阈值$d=500\text{\textrm{m}}$和空间参与度阈值$0.6$,
	对北京市微博POI类别进行同位模式挖掘,结果显示阶数越高越呈现商业聚集模式,
	其中最高六阶模式为(KTV,中餐厅,咖啡厅,甜品店,美容美发店,酒吧)。
	
	(3)根据全国新浪微博用户在$2016$年春节期间的空间位置数据,使用Spatial-Spark构建全国城市之间人口流动网络图。
	首先计算每个城市人口流入量、流出量和流入流出比,发现全国城市在春节期间人口流动呈现多样性;然后采用PageRank算法
	计算城市在人口流动网络图中的权重,发现城市权重与城市GDP发展的存在相关性,并根据权重将中心城市划分四个层次;
	最后对社群挖掘算法进行并行化改进,对人口流动网络图进行社群挖掘,发现城市联系紧密性与省份有关,地理位置对其
	影响很大,但也存在突破地理空间位置限制的城市。
	
	本文包含图\ref{totalfigure}幅,表\ref{totaltable}张,参考文献\ref{totalbib}篇。
	%--------中文关键词
	\CKeyWords{Spatial-Spark, 新浪微博, 同位模式, 网络分析}
\end{cabstract}

%--------英文摘要
\begin{eabstract}
	% Information plays a signifcant role for strategy making in moderen society. As the tremendous 
	% increment of internet, the social spatial data implying folk's behavior has been explosing. 
	% Howerver, the heterogeneous, irregular and countless data causes the spatial data's analysis,
	% mining and knowledge discovering becoming more difficult to handle.
	% As the tremendous development of mobile internet, the social networks, represented by Weibo, are 
	% releasing explosing amount of data. The spatial data in that is deserved of particular attention and 
	% reserarch. Howerver, the heterogeneous, irregular and countless data leads the spatial data's analysis,
	% mining and knowledge discovering becoming more intractable.
	Digit City plays a vital role in smart city, and facing explosing amount data fetch, management, analysis
	and mining challenges. Mobing internet developments cause the cyber spatial data increaing unbelievablely which
	contains much informations for building the smart cities. Howerver, the heterogeneous, irregular and countless data leads the spatial data's analysis,
	mining and knowledge discovering becoming more intractable.
	
	Traditional spatial analysis method cannot meet the above requires, so this thesis builds the parallel 
	spatial computing framwork, called Spatail-Spark, based on popular open-source parallel computing framwork.
	Using this framework, it mines the spatial co-location pattern in Weibo's POIs and population graph according to 
	the Weibo users's spatial locations. The achievements and conclusions of this thesis are listed as follows:
	
	(1) It expands the Spark RDD in spatial geometries called Spatial RDD, including point, line and polygon. 
	And those RDDs support the spaital data reading and writing, spatial coordinates converting and spatial 
	index building in separate partition. It also provides three spatial query modules: sptaial topology query,
	spatial K near neighbour query and spatial join query. At last, it designs experiments to figure out the 
	efficiency of Spatial-Spark by various comparsions.
	
	(2) After fetching out Weibo's POIs through Weibo's API, it conducts the spatial co-location pattern mining
	algorithm in those data. It analysis the vital key of spatial co-location pattern and redesigns the parallel 
	% algorithm for efficient perform using Sptial-Spark. It comes out that the patterns shows more commercial 
	% characteristics when the pattern's order is bigger on the condition that 
	% the spaital distance threshold is $500$ meters and the participating index threshold is $0.6$. Moreover, the 
	% six order pattern is (KTV, Chinese Restaurant, Coffe Bar, Dessert Stall, Barber Shop, Bar Pub).
	algorithm for efficient performance using Spatial-Spark. It comes out that the there are huge differences 
	among the different cities, taking the Shanghai, Wuhan and Chongqing. It also provides that the patterns shows more commercial 
	characteristics when the pattern's order is bigger on the condition that 
	the spaital distance threshold is $500$ meters and the participating index threshold is $0.6$. Moreover, the 
	six order pattern is (KTV, Chinese Restaurant, Coffe Bar, Dessert Stall, Barber Shop, Bar Pub).
	
	(3) It constructs the cities floating population graph using Spatial-Spark from the national Weibo's users 
	location in 2016 Chinese New Year holiday. At first, it calculates the popluation amount of flow-in and flow-out,
	and flow ratio, which displays the diversity of floating population among the national cities. Then it figures
	out the national cities' weights in floating population network under PageRank algorithm. It finds out the relationship
	between city's weight and city's development and divides the natianal cities into four levels. Lastly, it applys the 
	parallel community mining algorithm in this network and uncovers the phenomenon that the cities's popluaiton relationships are 
	almost limited by the provinces and there are also some exceptional cases. 
	
	This thesis contains \ref{totalfigure} figures, \ref{totaltable} tables and \ref{totalbib} references。
	%--------英文关键词
	\EKeyWords{Spatial-Spark, Weibo, Co-Location Pattern, Graph Analysis}
\end{eabstract}